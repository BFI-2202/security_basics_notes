\documentclass{article}
\usepackage[utf8]{inputenc}

\usepackage[T2A]{fontenc}
\usepackage[utf8]{inputenc}
\usepackage[russian]{babel}

\usepackage{tabularx}
\usepackage{amsmath}
\usepackage{pgfplots}
\usepackage{geometry}
\usepackage{multicol}
\geometry{
    left=1cm,right=1cm,top=2cm,bottom=2cm
}
\newcommand*\diff{\mathop{}\!\mathrm{d}}

\newtheorem{definition}{Определение}
\newtheorem{theorem}{Теорема}

\DeclareMathOperator{\sign}{sign}

\usepackage{hyperref}
\hypersetup{
    colorlinks, citecolor=black, filecolor=black, linkcolor=black, urlcolor=black
}

\title{Дискретная математика}
\author{Лисид Лаконский}
\date{September 2023}

\begin{document}
\raggedright

\maketitle

\tableofcontents
\pagebreak

\section{Лекция — 20.09.2023}

\textbf{Субъекты информационной безопасности}:

\begin{enumerate}
    \item Государство
    \item Общество
    \item Личность
\end{enumerate}

\textbf{Национальными интересами} в информационной сфере являются:

\begin{enumerate}
    \item обеспечение и защита конституционных прав и свобод человека и гражданина в части, касающейся получения и использования информации, неприкосновенности частной жизни при использовании информационных технологий, обеспечение информационной поддержки демократических институтов, механизмов взаимодействия государства и гражданского общества, а также применение информационных технологий в интересах сохранения культурных, исторических и духовно-нравственных ценностей многонационального народа Российской Федерации;
    \item обеспечение устойчивого и бесперебойного функционирования информационной инфраструктуры, в первую очередь критической информационной инфраструктуры Российской Федерации (далее - критическая информационная инфраструктура) и единой сети электросвязи Российской Федерации, в мирное время, в период непосредственной угрозы агрессии и в военное время;
    \item развитие в Российской Федерации отрасли информационных технологий и электронной промышленности, а также совершенствование деятельности производственных, научных и научно-технических организаций по разработке, производству и эксплуатации средств обеспечения информационной безопасности, оказанию услуг в области обеспечения информационной безопасности;
    \item доведение до российской и международной общественности достоверной информации о государственной политике Российской Федерации и ее официальной позиции по социально значимым событиям в стране и мире, применение информационных технологий в целях обеспечения национальной безопасности Российской Федерации в области культуры;
    \item содействие формированию системы международной информационной безопасности, направленной на противодействие угрозам использования информационных технологий в целях нарушения стратегической стабильности, на укрепление равноправного стратегического партнерства в области информационной безопасности, а также на защиту суверенитета Российской Федерации в информационном пространстве.
\end{enumerate}

\textbf{Основные исполнительные органы обеспечения информационной безопасности}:

\begin{enumerate}
    \item Межведомственная комиссия по защите Гостайны
    \item Федеральная служба по техническому и экспортному контролю (ФСТЭК)
    \item Федеральная служба безопасности (ФСБ)
\end{enumerate}

\end{document}